%
% Dokumentation
%

% Kapitel dem Inhaltsverzeichnis hinzufügen
\addcontentsline{toc}{chapter}{Änderungshistorie}			% 1. Argument {toc} = Table Of Contents
																% 2. Argument {chapter} = Ebene bestimmen!
																% 3. Argument {Name} = Name des Eintrags
\noindent
\chapter*{Änderungshistorie}
\begin{tabularx}{\textwidth}{|l|X|l|l|} \hline
       \textbf{Version}  	& \textbf{Name}   	& \textbf{Kapitel}  						& \textbf{Datum}\\ \hline
       0.1           		& Florens Hückstädt & \ref{s_erforderliche_hw_sw} 				& 17.01.2015\\
       0.2           		& Florens Hückstädt & \ref{c_eclipse}, \ref{s_startserver}, \ref{s_gradle}, \ref{s_config_db} 		& 18.01.2015\\
        \hline
\end{tabularx}



\thispagestyle{plain}

\chapter{Installationsanleitung}\label{c_installationsanleitung}
In diesem Kapitel werden die notwendigen Schritte erläutert die Anwendung zu installieren und zu starten. Dabei werden mehrere Alternativen aufgezeigt. Des Weiteren wird auf die erforderliche Hard- und Software eingegangen.


\section{Erforderliche Hard- und Software}\label{s_erforderliche_hw_sw}
Zum Betreiben der Anwendung ist ein \ac{JRE} bzw zum Entwickeln ein \ac{JDK} in der Version 8 erforderlich.
Dafür gelten folgende Systemvoraussetzungen laut \cite{oraclejsv}

\begin{description}
   \item[Windows]~\par
   \begin{itemize}

    \item Windows 8 (Desktop)
    \item Windows 7
    \item Windows Vista SP2
    \item Windows Server 2008 R2 SP1 (64-Bit)
    \item Windows Server 2012 (64 Bit)
    \item RAM: 128 MB
    \item Datenträgerkapazität: 124 MB für JRE; 2 MB für Java Update
    \item Prozessor: Mindestens Pentium 2 266 MHz-Prozessor
    \item Browser: Internet Explorer 9 und höher, Firefox, Chrome
   \end{itemize}
   
      \item[Mac OS X]~\par
      \begin{itemize}
    \item Intel-basierter Mac unter Mac OS X 10.8.3+, 10.9+
    \item Administratorberechtigungen für die Installation
    \item 64-Bit-Browser
      \end{itemize}
      
      Ein 64-Bit-Browser (Beispiele: Safari, Firefox oder Chrome) ist zur Ausführung von Oracle Java auf Mac OS X erforderlich.
      
   \item[Linux]~\par
   \begin{itemize}

    \item Oracle Linux 5.5+1
    \item Oracle Linux 6.x (32-Bit), 6.x (64-Bit)2
    \item Oracle Linux 7.x (64-Bit)2
    \item Red Hat Enterprise Linux 5.5+1, 6.x (32-Bit), 6.x (64-Bit)2
    \item Ubuntu Linux 12.04 LTS, 13.x
    \item Suse Linux Enterprise Server 10 SP2+, 11.x
    \item Browser: Firefox

   \end{itemize}
      
   
\end{description}

\section{Starten des Servers}\label{s_startserver}
Das Starten des Servers kann grundsätzlich über einen direkten java-Aufruf geschehen (siehe \ref{s_startserver_java}). Eine weitere Möglichkeit ist der Start des Servers über das Application-Plugin von gradle (siehe \ref{s_startserver_gradle}). Im Auslieferungszustand wird eine interne HSQLDB-Datenbank verwendet. Daher ist ein direktes Starten des Servers möglich. Zur Verwendung einer anderen Datenbank siehe Abschnitt \ref{s_config_db}.

\subsection{Starten des Servers über die Kommandozeile}\label{s_startserver_java}
Zum Starten des Servers über einen direkten Java-Aufruf mit der Kommandozeile sind folgende Schritte durchzuführen:
   \begin{enumerate}

    \item Navigation mit der Kommandozeile in das Root-Verzeichnis von weFactor und Ausführen folgenden Befehls:
    \begin{lstlisting}
    gradlew assemble
    \end{lstlisting}
    Dadurch wird im Verzeichnis /build/libs die Datei wefactor.jar erzeugt. Gradle ist im Projekt integriert und muss nicht separat installiert werden (siehe \ref{s_gradle}).
    \item Ausführen des jars über den Befehl:
    \begin{lstlisting}
    java -jar wefactor.jar
    \end{lstlisting}

   \end{enumerate}


\subsection{Starten des Servers über Gradle}\label{s_startserver_gradle}
Das Gradle Application-Plugin erweitert die Sprachen-Plugins um allgemeine anwendungsbezogene Tasks. Es erlaubt das Ausführen und Bündeln von Anwendungen für die \ac{JVM}.

Für weitere Informationen zur Verwendung von Gradle innerhalb des weFactor-Projeks siehe \ref{s_gradle}.

Für weitere Informationen zum Application-Plugin siehe
Gradle Dokumentation\footnote{\url{http://www.gradle.org/docs/current/userguide/application_plugin.html}}.

\subsubsection{Gradle-Plugin für Spring-Boot-Anwendungen}
Zum Starten des Servers über den Gradle-Task bootRun sind folgende Schritte durchzuführen:
   \begin{enumerate}

    \item Navigation mit der Kommandozeile in das Root-Verzeichnis von weFactor
    \item Ausführen des Gradle-Befehls:
    \begin{lstlisting}
    gradlew bootRun
    \end{lstlisting}

   \end{enumerate}

\subsubsection{Gradle distZip}
Der Task distZip erzeugt ein Distributionsarchiv mit den erforderlichen Bibliotheken und entsprechenden Startskripten. Zur Erzeugung der Distribution sind folgende Schritte notwendig:

   \begin{enumerate}

    \item Navigation mit der Kommandozeile in das Root-Verzeichnis von weFactor und Ausführen folgenden Befehls:
    \begin{lstlisting}
    gradlew distZip
    \end{lstlisting}
    Dadurch wird im Verzeichnis /build/distributions die Datei wefactor.zip erzeugt.
    \item Entpacken des zip-Archivs.
    \item Starten der Anwendung über das Startskript
    \begin{lstlisting}
    wefactor.bat
    \end{lstlisting}
    Das Startscript ist im bin-Verzeichnis zu finden.
        

   \end{enumerate}




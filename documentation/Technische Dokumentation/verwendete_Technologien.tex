
\thispagestyle{plain}

\chapter{Verwendete Technologien}\label{c_verTech}
sfdsfd

\section{Gradle}\label{s_gradle}
Als Build-Management-Werkzeug wurde in diesem Projekt Gradle verwendet. Für ausführliche Informationen siehe Gradle Dokumentation\footnote{\url{https://www.gradle.org/documentation}}

Folgende Plugins sind derzeit im Buildskript integriert:\\
\begin{tabularx}{\textwidth}{|l|X|} \hline
       \textbf{Plugin}  	& \textbf{Beschreibung} \\ \hline

       java           		& Das java-Plugin fügt dem Projekt Kompilierungs-, Test-Möglichkeiten hinzu.\\\hline
       eclipse           	& Das eclipse-Plugin generiert Dateien die von der Eclipse \ac{IDE} verwendet werden.\\\hline
	   spring-boot          & Das spring-boot-plugin erlaubt das Erstellen von ausführbaren Spring Boot jar- und war-Dateien. Außerdem ist es möglich Spring Boot-Anwendungen direkt zu starten.\\\hline
	   application          & Das Gradle Application-Plugin erweitert die Sprachen-Plugins um allgemeine anwendungsbezogene Tasks. Es erlaubt das Ausführen und Bündeln von Anwendungen für die \ac{JVM}.\\\hline
       jacoco           	& Jacoco bietet code coverage-Metriken für Java Code.\\\hline
        
\end{tabularx}

\section{Spring MVC}\label{s_springmvc}

\section{Thymeleaf}\label{s_thymeleaf}

\section{Twitter Bootstrap}\label{s_bootstrap}